\documentclass[a4j]{jarticle}
\title{進化的アルゴリズムレポート \ 課題1}
\author{情報科学類, 3年, 2クラス, 学籍番号:201311403 \ 山岡 洸瑛, Yamaoka
Kouei}
\date{\today}

%余白設定
\setlength{\topmargin}{20mm}
\addtolength{\topmargin}{-1in}
\setlength{\oddsidemargin}{20mm}
\addtolength{\oddsidemargin}{-1in}
\setlength{\evensidemargin}{15mm}
\addtolength{\evensidemargin}{-1in}
\setlength{\textwidth}{170mm}
\setlength{\textheight}{254mm}
\setlength{\headsep}{0mm}
\setlength{\headheight}{0mm}
\setlength{\topskip}{0mm}


\def\pgfsysdriver{pgfsys-dvipdfmx.def}

\usepackage{ascmac}
\usepackage{here}
\usepackage{tikz}
\usetikzlibrary{trees}
\thispagestyle{empty}
\usepackage{bm}

\usepackage{listings,jlisting}
\renewcommand{\lstlistingname}{リスト}
\lstset{
basicstyle=\ttfamily\scriptsize,
commentstyle=\textit,
classoffset=1,
keywordstyle=\bfseries,
frame=tRBl,
framesep=5pt,
showstringspaces=false,
numbers=left,
stepnumber=1,
numberstyle=\tiny,
tabsize=2
}


\begin{document}
\maketitle

\section*{グラフ色塗り問題の作成}
グラフ色塗り問題の作成のため,隣接行列を作成するプログラムを作成する.
0で初期化された$n*n$の2次元配列を作成し,その上三角要素にランダムに1を
代入することで隣接行列を作成する.\\\ \ \ 
ランダムに1を代入するため,上三角要素の数と同じ長さの配列を用意する.こ
こで上三角要素の数を$X$個とする.$0〜X$を範囲として乱数を生成,それをint
型にキャストすることで$0〜(X-1)$の範囲の整数が得られる.その整数番目の配
列の要素を1とすることを繰り返し規定の数1にする.ここで得た配列を順に上三
角要素にコピーすることで隣接行列を作成する.リスト\ref{prog}にプログラムの1部を示
す.ソース内のN[k]にはノードの数が格納されており,forループを使うことで1
度実行すれば良いようにしてある.また,作成した隣接行列はノード数が$30, 60, 90, 120, 150$の5つの密結合問題と疎
結合問題の計10個だが,大きすぎてレポートには示せないため,ここではノード
数を9として作成し表\ref{sparse}, 表\ref{dense}に示す.

\begin{lstlisting}[caption=makeAdjecencyMatrix.c, label=prog, xleftmargin=1cm]
int main(void){
    //seed 509
    srand(509);

    const int N[5] = {30, 60, 90, 120, 150};   // number if nodes
    int i, j, k; // for for loop
    int red, green;  // color range
    // ex)r:9, g:19, b:29(not use) @ N[k]=30 
    red = N[k] / 3 - 1;
    green = N[k] / 3 * 2 - 1;

    int SMat[N[k]][N[k]]; // AdjacencyMatrix for sparse
    int SLink = 3 * N[k]; // number of links for sparse

    // init
    for(i=0; i<N[k]; i++){
      for(j=0; j<N[k]; j++){
        SMat[i][j] = 0;
      }
    }

    // make links
    // calc number of diferent color elements
    int diffColorNum = pow(N[k], 2) / 3;

    // make array for upper tri elem and init
    int SUpperTri[diffColorNum]; // for sparse
    for(i=0; i<diffColorNum; i++){
      SUpperTri[i] = 0;
    }

    // choice random index for sparse
    int count = 0;
    int r;
    while(TRUE){
      r = diffColorNum * (rand()/(RAND_MAX+1.0));
      if(SUpperTri[r] != 1){
        SUpperTri[r] = 1;
        count++;
      }
      if(count == SLink){
        break;
      }
    }

    // input SUpperTri to Adjacency Matrix
    count = 0;
    for(i=0; i<N[k]; i++){
      for(j=0; j<N[k]; j++){
        if(i <= red && j > red || i > red && i <= green && j > green){
          SMat[i][j] = SUpperTri[count];
          count++;
        }
      }
    }
    return 0;
}
\end{lstlisting}

\begin{table}[htb]
 \begin{center}
  \begin{tabular}[tb]{|c|c|c|c|c|c|c|c|c|} \hline
0 & 0 & 0 & 1 & 1 & 1 & 1 & 1 & 1 \\ \hline
0 & 0 & 0 & 1 & 1 & 1 & 1 & 1 & 1 \\ \hline
0 & 0 & 0 & 1 & 1 & 1 & 1 & 1 & 1 \\ \hline
0 & 0 & 0 & 0 & 0 & 0 & 1 & 1 & 1 \\ \hline
0 & 0 & 0 & 0 & 0 & 0 & 1 & 1 & 1 \\ \hline
0 & 0 & 0 & 0 & 0 & 0 & 1 & 1 & 1 \\ \hline
0 & 0 & 0 & 0 & 0 & 0 & 0 & 0 & 0 \\ \hline
0 & 0 & 0 & 0 & 0 & 0 & 0 & 0 & 0 \\ \hline
0 & 0 & 0 & 0 & 0 & 0 & 0 & 0 & 0 \\ \hline
  \end{tabular}
  \caption{疎結合問題用隣接行列}
 \label{sparse}
 \end{center}
\end{table}

\begin{table}[htb]
 \begin{center}
  \begin{tabular}[tb]{|c|c|c|c|c|c|c|c|c|} \hline
0 & 0 & 0 & 1 & 1 & 1 & 1 & 0 & 0 \\ \hline
0 & 0 & 0 & 1 & 0 & 0 & 1 & 1 & 1 \\ \hline
0 & 0 & 0 & 0 & 0 & 0 & 1 & 1 & 0 \\ \hline
0 & 0 & 0 & 0 & 0 & 0 & 1 & 1 & 1 \\ \hline
0 & 0 & 0 & 0 & 0 & 0 & 1 & 1 & 1 \\ \hline
0 & 0 & 0 & 0 & 0 & 0 & 0 & 1 & 1 \\ \hline
0 & 0 & 0 & 0 & 0 & 0 & 0 & 0 & 0 \\ \hline
0 & 0 & 0 & 0 & 0 & 0 & 0 & 0 & 0 \\ \hline
0 & 0 & 0 & 0 & 0 & 0 & 0 & 0 & 0 \\ \hline
  \end{tabular}
 \end{center}
  \caption{密結合問題用隣接行列}
 \label{dense}
\end{table}

\end{document}
