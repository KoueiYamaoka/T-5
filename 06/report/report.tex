\documentclass[a4j]{jarticle}
\title{進化的アルゴリズムレポート \ 課題6}
\author{情報科学類, 3年, 2クラス, 学籍番号:201311403 \ 山岡 洸瑛, Yamaoka
Kouei}
\date{2016年1月13日}

%余白設定
\setlength{\topmargin}{20mm}
\addtolength{\topmargin}{-1in}
\setlength{\oddsidemargin}{20mm}
\addtolength{\oddsidemargin}{-1in}
\setlength{\evensidemargin}{15mm}
\addtolength{\evensidemargin}{-1in}
\setlength{\textwidth}{170mm}
\setlength{\textheight}{254mm}
\setlength{\headsep}{0mm}
\setlength{\headheight}{0mm}
\setlength{\topskip}{0mm}


\def\pgfsysdriver{pgfsys-dvipdfmx.def}

\usepackage{ascmac}
\usepackage{here}
\usepackage{tikz}
\usetikzlibrary{trees}
\thispagestyle{empty}

\usepackage{listings,jlisting}
\renewcommand{\lstlistingname}{リスト}
\lstset{
basicstyle=\ttfamily\scriptsize,
commentstyle=\textit,
classoffset=1,
keywordstyle=\bfseries,
frame=tRBl,
framesep=5pt,
showstringspaces=false,
numbers=left,
stepnumber=1,
numberstyle=\tiny,
tabsize=2
}

\begin{document}
\maketitle
\section*{最近隣法(NN法)と2optによる巡回セールスマン問題の解決}
\subsection*{NN法と2optのアルゴリズム}
アルゴリズムは与えられたものを使うため省略する.
\section*{実験}
実験結果を表\ref{result}に示す.最小値の項目にある括弧内には最適解を示している.また,2optにおいて,交換されるリンクが同一のリンクの繰り返しになってしまうことがあった.順回路を1回走査する毎に,改善された回数が3回7回3回7回...のようになってしまった.このような場合は改善しきれなかったとして扱い,表にその回数を示した.
標準偏差を平均で割った変動係数も表に示した.これより各問題において解のばらつきは同程度であることがわかる.
\begin{table}[H]
 \begin{center}
  \caption{実験結果}
  \label{result}
  \begin{tabular}[tb]{|c|c|c|c|c|c|c|} \hline
   & 最大値 & 最小値(最適解) & 平均値 & 標準偏差 & 変動係数 & 改善し切れなかった回数 \\ \hline
   eil51 & 492 & 433(426) & 454.8 & 12 & 0.0264 & 1 \\ \hline
   pr76 & 122746 & 110058(108159) & 115490.3 & 3005.6 & 0.0260 & 0 \\ \hline
   rat99 & 1426 & 1263(1211) & 1350.2 & 34.7 & 0.0257 & 0 \\ \hline
   kroA100 & 24794 & 21865(21282) & 22924.6 & 722.8 & 0.0315 & 1 \\ \hline
   ch130 & 7100 & 6304(6110) & 6667.5 & 150 & 0.0225 & 0 \\ \hline
  \end{tabular}
 \end{center}
\end{table}

\subsection*{間違いの修正}
2optで解を1回走査するたびに改善された回数が3回,7回,3回...などのようにループしてしまう問題を解決した.この問題の原因は,選択したリンク以外の部分も変更されてしまっていたことである.解は道順を示しており有向グラフになっている.そのため2optでノードを交換するだけでは道順が狂ってしまう.これを改善するために,交換するノードとノードを含め,その間を逆順にした.これらについてまとめ,図\ref{miss}に示す.また修正後の実験結果を改めて表\ref{re}に示す.

\begin{figure}[H]
 \begin{center}
  \includegraphics[width=20cm, clip, bb=-100 0 814 449]{miss.jpg}
  \caption{正しい2optのアルゴリズム}
  \label{miss}
 \end{center}
\end{figure}

\begin{table}[H]
 \begin{center}
  \caption{実験結果}
  \label{re}
  \begin{tabular}[tb]{|c|c|c|c|c|c|c|} \hline
 & 最大値 & 最小値(最適解) & 平均値 & 標準偏差 & 変動係数 \\ \hline
eil51 & 478 & 432(426) & 451.9 & 10.6 & 0.0235 \\\hline
pr76 & 124110 & 109307(108159) & 114470.9 & 3561.6 & 0.0311 \\\hline
rat99 & 1392 & 1261(1211) & 1322 & 29.9 & 0.0226 \\\hline
kroA100 & 24958 & 21376(21282) & 22841.4 & 622.7 & 0.0273 \\\hline
ch130 & 6999 & 6307(6110) & 6643.1 & 158.6 & 0.0239 \\\hline
  \end{tabular}
 \end{center}
\end{table}


\subsubsection*{ソースコード}
作成したソースコードをリスト\ref{NN2opt.c}に示す.ただし,tspファイルの読み込みなど重要でないと思われる部分などは省略している.なお.問題点は修正済みである.
\begin{lstlisting}[caption=NN2opt.c, label=NN2opt.c, xleftmargin=1cm]
int main(int argc, char *argv[]){

  // variable declaration
  int cityNum; // number of citis
  float x, y, x2, y2; // for tsp file
  int i, j, k, l; // for for loop
  int max, min;
  float ave, standardDeviation; // for output
  int count, tmp, sum, miss;
  int maxloop = 1000;

  // usage
  if(argc == 1 || strcmp(argv[1], "-help") == 0){
    printf("usage: $ ./NN2opt tspFileNumber\n");
    printf("tspFileNumber = 0:eil51, 1:pr76, 2:rat99, 3:kroA100, 4:ch130\n");
    return 0;
  }

  // input city locaions
  float X[cityNum], Y[cityNum]; // city locations
  for(i=0; i<cityNum; i++){
    X[i] = buffx[i];
    Y[i] = buffy[i];
  }
  // end input tsp file

  // variable declaration
  float eucArray[cityNum][cityNum];
  int check[cityNum]; // for check to passed city
  int solutions[cityNum][cityNum];
  int tourLength[cityNum];
  int c1, c2, c3, c4; // selected city
  int l1, l2, l3, l4; // length (c1, c2), (c3, c4), (c1, c3), (c2, c4)
  int total;

  // calc EUC_2D matrix
  // init diagonal 
  for(i=0; i<cityNum; i++){
    eucArray[i][i] = 0;
  }
  // calc 
  for(i=0; i<cityNum; i++){
    for(j=i+1; j<cityNum; j++){
      tmp = calcEuc(X[i], X[j], Y[i], Y[j]);
      eucArray[i][j] = tmp;
      eucArray[j][i] = tmp;
    }
  }

  // start NN and 2opt
  miss = 0;
  for(i=0; i<cityNum; i++){

    // init
    for(j=0; j<cityNum; j++){
      check[j] = 0;
    }
    
    // make solution by NN
    solutions[i][0] = i; // start city i
    check[i] = 1;
    for(j=1; j<cityNum; j++){
      min = NNmethod(check, eucArray[i], cityNum); // find min EUC_2D
      check[min] = 1; // check city
      solutions[i][j] = min;
    }

    // start 2opt
    for(l=0; l<maxloop; l++){
      count = 0;
      total = 0;
      for(j=0; j<cityNum-2; j++){ // link 1 = (c1, c2)
        c1 = solutions[i][j];
        c2 = solutions[i][j+1];
        for(k=j+2; k<cityNum; k++){ // link 2 = (c3, c4)
          c3 = solutions[i][k];             // if k = (cityNum - 1) 
          c4 = solutions[i][(k+1)%cityNum]; // then k+1 means 0 (= cityNum % cityNum)
          
          l1 = calcEuc(X[c1], X[c2], Y[c1], Y[c2]); // length c1, c2 
          l2 = calcEuc(X[c3], X[c4], Y[c3], Y[c4]); // length c3, c4
          l3 = calcEuc(X[c1], X[c3], Y[c1], Y[c3]); // length c1, c3
          l4 = calcEuc(X[c2], X[c4], Y[c2], Y[c4]); // length c2, c4
          
          if(l1 + l2 > l3 + l4){ // if before length > after length then swap
            reverse(solutions[i], j+1,  k);
            swap(&c2, &c3);
            count++;
          }
        }
      }
      total = count;
      if(count == 0) // if not swaped
        break;
      if(l == maxloop-1){
        printf("failed: start at %d\n", i);
        miss++;
      }
    }
  }
  // end NN and 2opt
  //calc tour length
  for(i=0; i<cityNum; i++){
    sum = 0;
    for(j=0; j<cityNum; j++){ // calc link(c1, c2) length
      c1 = solutions[i][j]; 
      c2 = solutions[i][(j+1)%cityNum];
      sum += calcEuc(X[c1], X[c2], Y[c1], Y[c2]);
    }
    tourLength[i] = sum;
    ave += sum;
    if(sum > max){
      max = sum;
    }
    if(sum < min){
      min = sum;
    }
  }
  ave /= cityNum;
  standardDeviation = calcStanDev(tourLength, cityNum, ave);
  printf("max = %d, min = %d, ave = %f, sd = %f, %d miss\n", max, min, ave, standardDeviation, miss);  
  return 0;
}

/* 
 *c; checkArray
 *euc: eucArray[i]
 num: cityNum
 return: index of min EUC_2D
 */
int NNmethod(int *c, float *euc, int num){
  int i, minNum;
  int m = INT_MAX;
  for(i=0; i<num; i++){
    if(*c == 0 && *euc != 0.0 && m > *euc){ // if (not checked and not 0 and min)
      m = *euc;
      minNum = i;
    }
    if(i<num-1){
      ++c;
      ++euc;
    }
  }
  return minNum;
}

/*
  x1, x2, y1, y2: num
  return: EUC_2D
*/
int calcEuc(float x1, float x2, float y1, float y2){
  return (int)(sqrt(pow(x1 - x2, 2) + pow(y1 - y2, 2)) + 0.5);
   
}

void swap(int *x, int *y){
  int tmp = *x;
  *x = *y;
  *y = tmp;
}

/*
  *array: calc array pointer
  length *array ength
  ave: calc array average
  return: standard_deviation
*/
float calcStanDev(int *array, int length, float ave){
  int i;
  float sum = 0;
  for(i=0; i<length; i++){
    sum += pow(*array - ave, 2);
    if(i<length-1)
      ++array;
  }
  sum /= length; // dispersion

  return sqrt(sum);
}

/* reverse *array from start to end */
void reverse(int *array, int start, int end){
  int i;
  for(i=0; i<(end-start+1)/2; i++){
    swap(&array[start + i], &array[end - i]);
  }
}
\end{lstlisting}


\end{document}
