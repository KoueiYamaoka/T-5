\documentclass[a4j]{jarticle}
\title{進化的アルゴリズムレポート \ 課題8}
\author{情報科学類, 3年, 2クラス, 学籍番号:201311403 \ 山岡 洸瑛, Yamaoka
Kouei}
\西暦
\date{2016年1月27日}

%余白設定
\setlength{\topmargin}{20mm}
\addtolength{\topmargin}{-1in}
\setlength{\oddsidemargin}{20mm}
\addtolength{\oddsidemargin}{-1in}
\setlength{\evensidemargin}{15mm}
\addtolength{\evensidemargin}{-1in}
\setlength{\textwidth}{170mm}
\setlength{\textheight}{254mm}
\setlength{\headsep}{0mm}
\setlength{\headheight}{0mm}
\setlength{\topskip}{0mm}


\def\pgfsysdriver{pgfsys-dvipdfmx.def}

\usepackage{ascmac}
\usepackage{here}
\usepackage{tikz}
\usetikzlibrary{trees}
\thispagestyle{empty}

\usepackage{listings,jlisting}
\renewcommand{\lstlistingname}{リスト}
\lstset{
basicstyle=\ttfamily\scriptsize,
commentstyle=\textit,
classoffset=1,
keywordstyle=\bfseries,
frame=tRBl,
framesep=5pt,
showstringspaces=false,
numbers=left,
stepnumber=1,
numberstyle=\tiny,
tabsize=2
}

\begin{document}
\maketitle

\section*{MMAS(MAX-MIN Ant System)による巡回セールスマン問題の解決}
\section*{実験}
\subsection*{パラメータの決定}
巡回セールスマン問題eil51に対してパラメータ$\beta$の最適値を求める.その他のパラメータは与えられたものを使うため省略する.パラメータの調整は,乱数の初期値を変えて10回計算し,その結果を用いて行った.表\ref{beta}に$\beta$を2から5まで変えた時の順回路長(最良解)と反復回数の平均値と標準偏差を示す.順回路長と反復回数の平均は共に小数点以下を切り捨てている.なお,シード値は{509, 521, 523, 541, 547, 557, 563, 569,571, 577}とした.
\begin{table}[H]
 \begin{center}
  \label{beta}
  \caption{各$\beta$毎の実験結果}
  \begin{tabular}[tb]{|c|c|c|c|c|}\hline
 &$ \beta = 2 $&$ \beta = 3 $&$ \beta = 4 $&$ \beta = 5$ \\\hline
巡回路長の平均 & 427 & 428 & 427 & 428 \\\hline
標準偏差 & 1.41 & 0.32 & 1.05 & 0.45 \\\hline
反復回数の平均 & 1700 & 1726 & 1877 & 1666 \\\hline
標準偏差 & 284.9 & 304.02 & 425.94 & 340.72 \\\hline
 \end{tabular}
 \end{center}
\end{table}
\par しかし,これだけでは傾向がよくわからなかったため,各シード値につき連続で2回ずつ,計20回計算し,その結果を表\ref{beta2}にまとめた.これより$\beta \geq 3$の時,標準偏差が小さくなっている.しかし$\beta = 2$の時もそれほど大きくなく,平均的にその他よりも良いため,$\beta = 2$とする.
\begin{table}[H]
 \begin{center}
  \label{beta2}
  \caption{20回の試行における各$\beta$毎の実験結果}
  \begin{tabular}[tb]{|c|c|c|c|c|}\hline
 &$ \beta = 2 $&$ \beta = 3 $&$ \beta = 4 $&$ \beta = 5$ \\\hline
巡回路長の平均 & 427 & 428 & 428 & 428 \\\hline
標準偏差 & 1.36 & 0.32 & 0.5 & 0.45 \\\hline
反復回数の平均 & 1708 & 1694 & 1752 & 1664 \\\hline
標準偏差 & 299.75 & 312.67 & 386.77 & 308.75 \\\hline
 \end{tabular}
 \end{center}
\end{table}

\par 以下にパラメータの一覧を示す.念の為,与えられた物も示した.
\begin{itemize}
 \item $\alpha = 1$
 \item $\beta = 2$
 \item アリの数$m$ = 都市数$n$
 \item $\rho = 0.02$
 \item $p_{best} = 0.05$
\end{itemize}

\subsection*{実験結果}
示したパラメータで実験を行った.その結果を表\ref{result}に示す.表には各問題を乱数の初期値を変えて10回行い,最良解の順回路長(括弧内は最適解)と反復回数の平均値と標準偏差を示している.順回路長と反復回数の平均は共に小数点以下を切り捨てている.また,順回路長については,標準偏差を平均値で割った変動係数も示した.これにより,pr76が少し大きいが,大まかには全ての問題で標準偏差の大きさが揃っていると言える.

\begin{table}[H]
 \begin{center}
  \caption{実験結果}
  \label{result}
  \begin{tabular}[tb]{|c|c|c|c|c|c|} \hline
 & eil51 & pr76 & rat99 & kroA100 & ch130 \\\hline
巡回路長の平均 & 427(426) & 109368(108159) & 1212(1211) & 21361(21282) & 6174(6110) \\\hline
標準偏差 & 1.41 & 913.14 & 1.76 & 37.49 & 32.55 \\\hline
変動係数 & 0.0033 & 0.0083 & 0.0015 & 0.0018 & 0.0053 \\\hline
反復回数の平均 & 1700 & 2091 & 1757 & 1935 & 2391 \\\hline
標準偏差 & 284.9 & 281.07 & 189.1 & 110.08 & 402.36 \\\hline
  \end{tabular}
 \end{center}
\end{table}

\subsection*{最良解の順回路長の推移}
各問題の最良解の順回路長の推移のグラフを図\ref{00}から図\ref{44}に示す.なお,シード値は10回の試行における最良解を出したものとした.

\subsubsection*{・eil51}
シード値は523.最適解426に対して,最良解は426となり,最適解と一致した.
\begin{figure}[H]
 \begin{center}
  \includegraphics[width=15cm, clip, bb=-75 0 446 265]{pic/00.jpg}
  \caption{eil51の最良解の推移}
  \label{00}
 \end{center}
\end{figure}
\clearpage
\subsubsection*{・pr76}
シード値は521.最適解108159に対して,最良解は111464.
\begin{figure}[H]
 \begin{center}
  \includegraphics[width=15cm, clip, bb=-75 0 454 264]{pic/11.jpg}
  \caption{pr76の最良解の推移}
  \label{11}
 \end{center}
\end{figure}
\subsubsection*{・rat99}
シード値は547.最適解1211に対して,最良解は1213.
\begin{figure}[H]
 \begin{center}
  \includegraphics[width=15cm, clip, bb=-75 0 441 266]{pic/22.jpg}
  \caption{rat99の最良解の推移}
  \label{22}
 \end{center}
\end{figure}
\clearpage
\subsubsection*{・kroA100}
シード値は577.最適解21282に対して,最良解は21330.
\begin{figure}[H]
 \begin{center}
  \includegraphics[width=15cm, clip, bb=-75 0 444 266]{pic/33.jpg}
  \caption{kroA100の最良解の推移}
  \label{33}
 \end{center}
\end{figure}
\subsubsection*{・ch130}
シード値は509.最適解6110に対して,最良解は6207.
\begin{figure}[H]
 \begin{center}
  \includegraphics[width=15cm, clip, bb=-75 0 455 266]{pic/44.jpg}
  \caption{ch130の最良解の推移}
  \label{44}
 \end{center}
\end{figure}



\clearpage
\subsection*{ソースコード}
リスト1にソースコードを示す.なお,変数宣言や,巡回セールスマン問題の読み込みなど,重要でないと思われる部分は省略している.
\begin{lstlisting}[caption=MMAS.c, label=MMAS, xleftmargin=1cm]
int main(int argc, char *argv[]){
  const int seeds[10] = {509, 521, 523, 541, 547, 557, 563, 569, 571, 577}; // seeds
  int b = 0;
  int sdc = 0;
  
  // variable declaration
  int cityNum; // number of citis
  const int maxloop = 1000;

  srand(seeds[sdc]); 
  // variable declaration
  int bsf[cityNum]; // best-so-fat
  int solutions[m][cityNum];
  double tau[cityNum][cityNum]; // pheromone
  double p[m][cityNum]; // selection probability
  int tourLength[cityNum];
  int c1, c2; // selected city
  int bs, ibs; // tour length @best solution, @iteration

  // calc EUC_2D
  // init diagonal 
  for(i=0; i<cityNum; i++){
    eucArray[i][i] = 0;
  }
  // calc 
  for(i=0; i<cityNum; i++){
    for(j=i+1; j<cityNum; j++){
      tmp = calcEuc(X[i], X[j], Y[i], Y[j]);
      eucArray[i][j] = tmp;
      eucArray[j][i] = tmp;
    }
  }

  // init pheromone
  // make solution by NN
  solutions[0][0] = 0; // start city 0
  check[0][0] = 1;
  min = 0;
  for(i=1; i<cityNum; i++){
    min = NNmethod(check[0], eucArray[min], cityNum); // find min EUC_2D
    check[0][min] = 1; // check city
    bsf[i] = min;
  }
  
  // clac tour length 
  sum = 0;
  for(i=0; i<cityNum; i++){ // calc link(c1, c2) length
    c1 = bsf[i]; 
    c2 = bsf[(i+1)%cityNum];
    sum += eucArray[c1][c2];
  }
  bs = sum;
  // set initial pheromone
  taumax = 1.0 / (bs * rho);
  taumin = calcTauMin(taumax, cityNum, pbest);
  for(i=0; i<cityNum; i++){
    for(j=i+1; j<cityNum; j++){
      tau[i][j] = taumax;
      tau[j][i] = taumax;
    }
  }
  // end init pheromone
  
  // start iteration
  count = 0;
  loopcount = 0;
  while(1){
    loopcount++;
    for(i=0; i<m; i++){
      for(j=0; j<cityNum; j++){
        check[i][j] = 0;
      }
      tourLength[i] = 0;
    }
    // loop for ant
    for(ant=0; ant<m; ant++){
      c1 = cityNum * (rand()/(RAND_MAX+1.0)); // select first city
      check[ant][c1] = 1;
      solutions[ant][0] = c1;
      
      // loop for make tour
      for(city=1; city<cityNum; city++){
        // init
        sum = 0;
        for(i=0; i<m; i++){
          for(j=0; j<cityNum; j++){
            p[i][j] = 0;
          }
        }

        // calc selection probability ( p[][] )
        for(i=0; i<cityNum; i++){ // calc denominator
          if(check[ant][i] == 0){
            sum += pow(tau[c1][i], alpha) * pow((1.0 / eucArray[c1][i]), beta[b]);
          }
        }
        for(i=0; i<cityNum; i++){
          if(check[ant][i] == 0 && tau[c1][i] != 0.0){
            ftmp = pow(tau[c1][i], alpha) * pow((1.0 / eucArray[c1][i]), beta[b]) / sum;
            if(isnan(ftmp)){
              ftmp = 1.0 / (bs*rho);
            }
            p[c1][i] = ftmp;
            p[i][c1] = ftmp;
          }
        }
        // make selection probability array
        for(i=1; i<cityNum; i++){
          p[c1][i] += p[c1][i-1];
        }

        // select next city
        ftmp = p[c1][cityNum-1] * (rand()/(RAND_MAX+1.0));
        i = 0;
        while(1){
          if(p[c1][i] >= ftmp){
            c2 = i; // now c2 = the city next to c1
            check[ant][c2] = 1; // check
            solutions[ant][city] = c2; // input solution
            tourLength[ant] += eucArray[c1][c2]; // add length(c1, c2);
            c1 = c2;
            break;
          }
          i++;
        }
      }
      // end loop for make tour
      tourLength[ant] += eucArray[solutions[ant][0]][solutions[ant][cityNum-1]];
    }
    // end loop for ant

    // search iteration beat
    ibs = INT_MAX;
    for(i=0; i<m; i++){
      if(ibs > tourLength[i]){
        ibs = tourLength[i];
        minNum = i;
      }
    }
    
    // end decision
    if(bs <= ibs){ // continue 
      count++;
    }
    else{ // reset count and update best solution, taumax, and taumin, 
      bs = ibs;
      count = 0;
      for(i=0; i<cityNum; i++){ 
        bsf[i] = solutions[minNum][i]; // update best-so-far
      }
    }
    tl[loopcount-1] = bs;
    if(count == maxloop){// end
      break;
    }
    
    // update pheromone
    taumax = 1.0 / (rho * ibs);
    taumin = calcTauMin(taumax, cityNum, pbest);
    // evaporation
    for(i=0; i<m; i++){
      for(j=0; j<cityNum; j++){
        tau[i][j] *= (1.0 - rho);
      }
    }
    // add pheromone
    ftmp = 1.0 / ibs;
    for(i=0; i<cityNum; i++){
      c1 = solutions[minNum][i];
      c2 = solutions[minNum][(i+1)%cityNum];
      tau[c1][c2] += ftmp;
      tau[c2][c1] += ftmp;
    }
    // MAX-MIN
    for(i=0; i<cityNum; i++){
      for(j=i+1; j<cityNum; j++){
        if(tau[i][j] > taumax){
          tau[i][j] = taumax;
          tau[j][i] = taumax;
        }
        if(tau[i][j] < taumin){
          tau[i][j] = taumin;
          tau[j][i] = taumin;
        }
      }
    }
    // end update pheromone
  }
  return 0;
}

/* 
 *c; checkArray
 *euc: eucArray[i]
 num: cityNum
 return: index of min EUC_2D
 */
int NNmethod(int *c, int *euc, int num){
  int i, minNum;
  int m = INT_MAX;
  for(i=0; i<num; i++){
    if(*c == 0 && *euc != 0 && m > *euc){ // if (not checked and not 0 and min)
      m = *euc;
      minNum = i;
    }
    if(i<num-1){
      ++c;
      ++euc;
    }
  }
  return minNum;
}

/*
  x1, x2, y1, y2: num
  return: EUC_2D
*/
int calcEuc(double x1, double x2, double y1, double y2){
  return (int)(sqrt(pow(x1 - x2, 2) + pow(y1 - y2, 2)) + 0.5);
   
}

double calcTauMin(double taumax, int n, double p){
  double tmp = pow(p, (1.0 / n));
  return ((1.0 - tmp) * taumax) / ((n/2.0 - 1.0) * tmp);
}
\end{lstlisting}


\end{document}
