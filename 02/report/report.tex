\documentclass[a4j]{jarticle}
\title{進化的アルゴリズムレポート \ 課題2}
\author{情報科学類, 3年, 2クラス, 学籍番号:201311403 \ 山岡 洸瑛, Yamaoka
Kouei}
\西暦
\date{\today}

%余白設定
\setlength{\topmargin}{20mm}
\addtolength{\topmargin}{-1in}
\setlength{\oddsidemargin}{20mm}
\addtolength{\oddsidemargin}{-1in}
\setlength{\evensidemargin}{15mm}
\addtolength{\evensidemargin}{-1in}
\setlength{\textwidth}{170mm}
\setlength{\textheight}{254mm}
\setlength{\headsep}{0mm}
\setlength{\headheight}{0mm}
\setlength{\topskip}{0mm}


\def\pgfsysdriver{pgfsys-dvipdfmx.def}

\usepackage{ascmac}
\usepackage{here}
\usepackage{tikz}
\usetikzlibrary{trees}
\thispagestyle{empty}
\usepackage{bm}

\usepackage{listings,jlisting}
\renewcommand{\lstlistingname}{リスト}
\lstset{
basicstyle=\ttfamily\scriptsize,
commentstyle=\textit,
classoffset=1,
keywordstyle=\bfseries,
frame=tRBl,
framesep=5pt,
showstringspaces=false,
numbers=left,
stepnumber=1,
numberstyle=\tiny,
tabsize=2
}

\begin{document}
\maketitle

\section*{山登り法によるグラフ色塗り問題の解決}
\subsection*{HCアルゴリズム}
HCアルゴリズムを実装した.アルゴリズムは与えられた資料と同じである.
プログラムをレポートの最後,リスト\ref{HC.c}に示す.なお,プログラムは1
部のみ示し,変数宣言の1部や,隣接行列の読み込みなどは省略している.

\subsection*{IHC}
IHCのアルゴリズムを実装した.IHCは
山登り法が失敗した場合にもう一度山登り法を実行する,つまりリスト
\ref{HC.c}を成功するまで,もしくは規定回数繰り返すだけなので,ソースコー
ドは省略する.

\section*{実験}
\subsection*{実験条件(各種パラメータ)}
以下に各種パラメータの値を示す.なお,乱数のシード値は509とした.

\subsubsection*{問題サイズ(ノード数)}
問題サイズは疎結合,密結合共に30, 60, 90, 120, 150の5つとした.

\subsubsection*{失敗と判断するまでの繰り返し回数}
失敗と判断するまでの繰り返し回数は表\ref{maxloop}とした.これはそれぞれの問題
サイズについて,十分大きな回数繰り返すHCアルゴリズムを実行し,失敗と判断
しても問題ない回数を確認し,決定したものである.失敗と判断しても問題ない
回数は,成功した場合の繰り返し回数を複数回確認し決定した.確認のために失
敗時はその旨を,成功時にはその時の繰り返し回数と,リスタートした回数を出
力させた.成功時の出力をまとめた表を表\ref{loops}, \ref{loopd}に示す.な
お表\ref{loops}, \ref{loopd}
において,サイズNの疎結合問題を「疎(N)」,同様にサイズNの密結合問題を「密(N)」と
表し.繰り返し回数を「回数」,リスタート回数を「RE」としている.

\begin{table}[htb]
 \begin{center}  
  \begin{tabular}[tb]{|c|c|c|c|c|c|} \hline
   問題サイズ& 30& 60& 90& 120& 150 \\ \hline
   最大繰り返し回数(疎結合)& 200& 400& 800& 1000& 2000 \\ \hline
   最大繰り返し回数(密結合)& 150& 300& 400& 600& 800 \\ \hline
  \end{tabular}
  \caption{失敗と判断するまでの繰り返し回数}
  \label{maxloop}
 \end{center}
\end{table}

\begin{table}[htb]
 \begin{center}
  \begin{tabular}[tb]{|c||c|c||c|c||c|c||c|c||c|c|} \hline
   & \multicolumn{2}{|c||}{疎(30)} & \multicolumn{2}{|c||}{疎(60)} &
   \multicolumn{2}{|c||}{疎(90)} & \multicolumn{2}{|c||}{疎(120)} &
   \multicolumn{2}{|c|}{疎(150)}  \\   \cline{2-11}
   & 回数& RE & 回数& RE & 回数& RE & 回数& RE & 回数& RE  \\ \hline \hline
   成功1回目& 93& 1& 751& 11& 359& 11& 718& 124& 2980& 19\\ \hline
   成功2回目& 181& 2& 252& 13& 621& 34& 910& 196& 1827& 24\\ \hline
   成功3回目& 43& 3& 153& 16& 380& 47& 239& 342& 1528& 88\\ \hline
   成功4回目& 77& 4& 399& 26& 788& 69& 606& 392& 4636& 174\\ \hline
   成功5回目& 59& 5& 165& 34& 389& 83& 1092& 668& 1235& 640\\ \hline
   成功6回目& 383& 6& 717& 39& 675& 95& -& -& -& -\\ \hline
   成功7回目& 41& 7& 256& 46& 237& 96& -& -& -& -\\ \hline
   成功8回目& 77& 8& 330& 50& 669& 98& -& -& -& -\\ \hline
   成功9回目& 52& 9& 397& 54& -& -& -& -& -& -\\ \hline
   成功10回目& 192& 10& 279& 56& -& -& -& -& -& -\\ \hline
  \end{tabular}
  \caption{成功時の出力(疎結合問題)}
  \label{loops}
 \end{center}
\end{table}

\begin{table}[htb]
 \begin{center}
  \begin{tabular}[tb]{|c||c|c||c|c||c|c||c|c||c|c|} \hline
& \multicolumn{2}{|c||}{密(30)} & \multicolumn{2}{|c||}{密(60)} &
   \multicolumn{2}{|c||}{密(90)} & \multicolumn{2}{|c||}{密(120)} &
   \multicolumn{2}{|c|}{密(150)}  \\   \cline{2-11}
   & 回数& RE & 回数& RE & 回数& RE & 回数& RE & 回数& RE  \\ \hline \hline
   成功1回目& 85& 0& 172& 0& 267& 0& 402& 0& 587& 0\\ \hline
   成功2回目& 117& 1& 172& 1& 282& 1& 387& 1& 730& 1\\ \hline
   成功3回目& 79& 2& 174& 2& 276& 2& 448& 2& 656& 2\\ \hline
   成功4回目& 58& 3& 238& 3& 324& 3& 578& 3& 622& 3\\ \hline
   成功5回目& 103& 4& 128& 4& 258& 4& 328& 4& 649& 4\\ \hline
   成功6回目& 114& 5& 225& 5& 267& 5& 414& 5& 610& 5\\ \hline
   成功7回目& 68& 6& 156& 6& 362& 6& 505& 6& 484& 6\\ \hline
   成功8回目& 90& 7& 191& 7& 304& 7& 403& 7& 477& 7\\ \hline
   成功9回目& 106& 8& 168& 8& 216& 8& 404& 8& 648& 8\\ \hline
   成功10回目& 67& 9& 273& 9& 373& 9& 348& 9& 729& 9\\ \hline
  \end{tabular}
  \caption{成功時の出力(密結合問題)}
  \label{loopd}
 \end{center}
\end{table}

\subsubsection*{リスタートさせる回数}
IHCアルゴリズムでHCアルゴリズムを繰り返す回数について示す.表
\ref{loops}, \ref{loopd}からわかる通り,
疎結合問題で問題サイズが2桁の場合,リスタートさせる回数は,100回程度で十
分である.問題サイズが150程度の大きさになると500回ほど計算したほうが良い
だろう.密結合の場合は問題サイズが150程度であれば10回も必要ない.実験においてリスタートした回数を表
\ref{restart}に示す.

\begin{table}[htb]
 \begin{center}
  \begin{tabular}[t]{|c|c|c|c|c|c|} \hline
   問題サイズ& 30& 60& 90& 120& 150 \\ \hline
   リスタートする回数(疎結合)& 20& 50& 100& 500& 500 \\ \hline
   リスタートする回数(密結合)& 10& 10& 10& 10& 10\\ \hline
  \end{tabular}
  \caption{リスタートする回数}
  \label{restart}
 \end{center}
\end{table}

\subsection*{実験結果}
以下に上記パラメータで行った実験の結果を示す.表\ref{result}に成功/失敗
の表を示す.ただし,失敗と判断するまでの繰り返し回数を表\ref{loops},
\ref{loopd}によって求め,表\ref{maxloop}の値を用いたため,実行結果は表
\ref{loops}, \ref{loopd}とは異なっている.以下にそれぞれのパターンについ
て成功したタイミングとそのグラフを示す.

\begin{table}[htb]
 \begin{center}
  \begin{tabular}[tb]{|c|c|c|c|c|c|} \hline
   問題サイズ & 30& 60& 90& 120& 150 \\ \hline
   成功/失敗(疎結合)& 成功& 成功& 成功& 成功& 成功 \\ \hline
   成功/失敗(密結合)& 成功& 成功& 成功& 成功& 成功 \\ \hline
  \end{tabular}
  \caption{実験結果(成功/失敗)}
  \label{result}
 \end{center}
\end{table}

\subsubsection*{疎結合(問題サイズ30)}
2回目のHCアルゴリズムの125ステップ目に成功した.最大繰り返し回数は200回
なので,世代は325である.図は図\ref{30s}に示す.
\begin{figure}[htb]
 \begin{center}
  \includegraphics[width=14cm, clip, bb=-250 0 586 345]{violation30s.jpg}
  \caption{疎結合問題(30)の探索過程(繰り返し山登り法)}
  \label{30s}
 \end{center}
\end{figure}

\subsubsection*{疎結合(問題サイズ60)}
5回目のHCアルゴリズムの243ステップ目に成功した.最大繰り返し回数は400回
なので,世代は1843である.図は図\ref{60s}に示す.
\begin{figure}[htb]
 \begin{center}
  \includegraphics[width=14cm, clip, bb=-250 0 590 348]{violation60s.jpg}
  \caption{疎結合問題(60)の探索過程(繰り返し山登り法)}
  \label{60s}
 \end{center}
\end{figure}

\newpage
\subsubsection*{疎結合(問題サイズ90)}
46回目のHCアルゴリズムの189ステップ目に成功した.最大繰り返し回数は800回
なので,世代は36189である.図は図\ref{90s}に示す.
\begin{figure}[htb]
 \begin{center}
  \includegraphics[width=14cm, clip, bb=-250 0 581 343]{violation90s.jpg}
  \caption{疎結合問題(90)の探索過程(繰り返し山登り法)}
  \label{90s}
 \end{center}
\end{figure}

\subsubsection*{疎結合(問題サイズ120)}
96回目のHCアルゴリズムの455ステップ目に成功した.最大繰り返し回数は1000回
なので,世代は95445である.図は図\ref{120s}に示す.
\begin{figure}[htb]
 \begin{center}
  \includegraphics[width=14cm, clip, bb=-250 0 581 342]{violation120s.jpg}
  \caption{疎結合問題(120)の探索過程(繰り返し山登り法)}
  \label{120s}
 \end{center}
\end{figure}
\newpage
\subsubsection*{疎結合(問題サイズ150)}
80回目のHCアルゴリズムの1573ステップ目に成功した.最大繰り返し回数は2000回
なので,世代は159573である.図は図\ref{150s}に示す.

\begin{figure}[htb]
 \begin{center}
  \includegraphics[width=14cm, clip, bb=-250 0 588 341]{violation150s.jpg}
  \caption{疎結合問題(150)の探索過程(繰り返し山登り法)}
  \label{150s}
 \end{center}
\end{figure}

\subsubsection*{密結合(問題サイズ30)}
HCアルゴリズムの85ステップ目に成功した.図は図\ref{30d}に示す.
\begin{figure}[htb]
 \begin{center}
  \includegraphics[width=14cm, clip, bb=-250 0 587 343]{violation30d.jpg}
  \caption{密結合問題(30)の探索過程(山登り法)}
  \label{30d}
 \end{center}
\end{figure}
\newpage
\subsubsection*{密結合(問題サイズ60)}
HCアルゴリズムの172ステップ目に成功した.図は図\ref{60d}に示す.
\begin{figure}[htb]
 \begin{center}
  \includegraphics[width=14cm, clip, bb=-250 0 592 340]{violation60d.jpg}
  \caption{密結合問題(60)の探索過程(山登り法)}
  \label{60d}
 \end{center}
\end{figure}

\subsubsection*{密結合(問題サイズ90)}
HCアルゴリズムの267ステップ目に成功した.図は図\ref{90d}に示す.
\begin{figure}[htb]
 \begin{center}
  \includegraphics[width=14cm, clip, bb=-250 0 589 342]{violation90d.jpg}
  \caption{密結合問題(90)の探索過程(山登り法)}
  \label{90d}
 \end{center}
\end{figure}
\newpage
\subsubsection*{密結合(問題サイズ120)}
HCアルゴリズムの402ステップ目に成功した.図は図\ref{120d}に示す.
\begin{figure}[htb]
 \begin{center}
  \includegraphics[width=14cm, clip, bb=-250 0 587 347]{violation120d.jpg}
  \caption{密結合問題(120)の探索過程(山登り法)}
  \label{120d}
 \end{center}
\end{figure}

\subsubsection*{密結合(問題サイズ150)}
HCアルゴリズムの587ステップ目に成功した.図は図\ref{150d}に示す.
\begin{figure}[htb]
 \begin{center}
  \includegraphics[width=14cm, clip, bb=-250 0 586 341]{violation150d.jpg}
  \caption{密結合問題(150)の探索過程(山登り法)}
  \label{150d}
 \end{center}
\end{figure}

\subsection*{ソースコード}
\begin{lstlisting}[caption=HC.c, label=HC.c, xleftmargin=1cm]
#define RED 0
#define GREEN 1
#define BLUE 2

int main(void){

  // seed 509
  srand(509);

  const int size[5] = {30, 60, 90, 120, 150}; // number of nodes
  const int N = size[0]; // which to use? NN[0] to NN[4] or something int value
  const int maxloop = 200; // the maximum number of iterations for HC algorithm

  int graph[N][N]; // graph
  int solution[N];  // solution candidates
  int violation = 0; // violation points
  int violationArray[N]; // if violation=1 then 1 else 0
  int Rv = 0;  // RED violation
  int Gv = 0;  // GREEN violation
  int Bv = 0;  // BLUE violation
  int changeVar = 0; // change value

  ////// HC algorithm
  // make initial value(input RED or GREEN or BLUE)
  for(i=0; i<N; i++){
    solution[i] = 3.0 * (rand()/(RAND_MAX+1.0));
  }
  
  // hill climbing start
  for(k=0; k<maxloop; k++){
    // init  violation and violationArray
    violation = 0;
    for(i=0; i<N; i++){
      violationArray[i] = 0;
    }

    // calc Violation points and updaate violationArray
    for(i = 0; i < N; i++){
      for(j = i + 1; j < N; j++){
        if(graph[i][j] == 1 && solution[i] == solution[j]){
          violation++;
          violationArray[i] = 1;
          violationArray[j] = 1;
        }        
      }
    }
    
    // end determination
    if(violation == 0){
      loopCount++;
      printf("completed!! %d loops\n", k);
      return 0;
    }
    
    //// change variable
    changeVar = 0;
    // select variable, ex)[0,1,1,0,0,1]
    count = 0;
    for(i = 0; i < N; i++){
      if(violationArray[i] == 1){ // pick up violation variable, ex)[1,1,1]
        count++;
      }
    }
    // select random, ex)if changeVar = 1 then [1, changeVar, 1] 
    changeVar = (double)count * (rand()/(RAND_MAX+1.0)); 

    count = -1;   // becouse of origin 0. After first count++ , count = 0;
    for(i = 0; i < N; i++){
      if(violationArray[i] == 1){
        count++;
      }
      if(count == changeVar){
        changeVar = i;  // input changeVar to violationArray, ex)[0,1,changeVar,0,0,1]
        break;
      }
    }
    
    // if color is change to XXX then violation is ??? point
    Rv = 0; // XXX = RED, ??? = Rv
    Gv = 0; // XXX = GREEN, ??? = Gv
    Bv = 0; // XXX = BLUE, ??? = Bv
    for(i = 0; i < N; i++){
      if(graph[changeVar][i] == 1){ // if there is link (column)
        if(RED == solution[i]){ // if RED
          Rv++;
        }
        else if(GREEN == solution[i]){ // if GREEN
          Gv++;
        }
        else if(BLUE == solution[i]){ // if BLUE
          Bv++;
        }
      }
      if(graph[i][changeVar] == 1){ // if there is link (row)
        if(RED == solution[i]){ // if RED
          Rv++;
        }
        else if(GREEN == solution[i]){ // if GREEN
          Gv++;
        }
        else if(BLUE == solution[i]){ // if BLUE
          Bv++;
        }
      }
    }
    
    //change value (use min violation coler)(if same then random)
    if(Rv < Bv){
      if(Rv < Gv)
        solution[changeVar] = RED;
      else if(Gv < Rv)
        solution[changeVar] = GREEN;
      else if(Gv == Rv){ // RED = 0 or GREEN = 1
        solution[changeVar] = 2.0 * rand()/(RAND_MAX+1.0); // (double)[0,2) -> (int)0 or 1
      }
    }
    else if(Bv < Rv){
      if(Bv < Gv)
        solution[changeVar] = BLUE;
      else if(Gv < Bv)
        solution[changeVar] = GREEN;
      else if(Gv == Bv){ // GRENN = 1 or BLUE = 2
        solution[changeVar] = (2.0 * rand()/(RAND_MAX+1.0)) + 1.0; // (double)[0,2) + 1.0 -> (int)1 or 2
      }
    }
    else if(Rv == Bv){
      if(Gv < Rv)
        solution[changeVar] = GREEN;
      else if(Rv < Gv){ // RED = 0 or BLUE = 2
        r = 2.0 * rand()/(RAND_MAX+1.0); // (double)[0,2) -> (int)0 or 1
        if(r == 1){ // 0 or 1 -> if r=1 then r=2 -> 0 or 2
          r = 2;  // 0 or 2
        }
        solution[changeVar] = r;
      }
      else if(Rv == Gv){ // RED = 0 or GREEN = 1 or BLUE = 2
        solution[changeVar] = 3.0 * (rand()/(RAND_MAX+1.0)); // (double)[0,3) -> (int)0 or 1 or 2
      }
    }
    // end 1 step
  }
  return 0;
}
\end{lstlisting}

\end{document}
